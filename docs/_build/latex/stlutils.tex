%% Generated by Sphinx.
\def\sphinxdocclass{report}
\documentclass[letterpaper,10pt,english]{sphinxmanual}
\ifdefined\pdfpxdimen
   \let\sphinxpxdimen\pdfpxdimen\else\newdimen\sphinxpxdimen
\fi \sphinxpxdimen=.75bp\relax
\ifdefined\pdfimageresolution
    \pdfimageresolution= \numexpr \dimexpr1in\relax/\sphinxpxdimen\relax
\fi
%% let collapsible pdf bookmarks panel have high depth per default
\PassOptionsToPackage{bookmarksdepth=5}{hyperref}

\PassOptionsToPackage{warn}{textcomp}
\usepackage[utf8]{inputenc}
\ifdefined\DeclareUnicodeCharacter
% support both utf8 and utf8x syntaxes
  \ifdefined\DeclareUnicodeCharacterAsOptional
    \def\sphinxDUC#1{\DeclareUnicodeCharacter{"#1}}
  \else
    \let\sphinxDUC\DeclareUnicodeCharacter
  \fi
  \sphinxDUC{00A0}{\nobreakspace}
  \sphinxDUC{2500}{\sphinxunichar{2500}}
  \sphinxDUC{2502}{\sphinxunichar{2502}}
  \sphinxDUC{2514}{\sphinxunichar{2514}}
  \sphinxDUC{251C}{\sphinxunichar{251C}}
  \sphinxDUC{2572}{\textbackslash}
\fi
\usepackage{cmap}
\usepackage[T1]{fontenc}
\usepackage{amsmath,amssymb,amstext}
\usepackage{babel}



\usepackage{tgtermes}
\usepackage{tgheros}
\renewcommand{\ttdefault}{txtt}



\usepackage[Bjarne]{fncychap}
\usepackage{sphinx}

\fvset{fontsize=auto}
\usepackage{geometry}


% Include hyperref last.
\usepackage{hyperref}
% Fix anchor placement for figures with captions.
\usepackage{hypcap}% it must be loaded after hyperref.
% Set up styles of URL: it should be placed after hyperref.
\urlstyle{same}

\addto\captionsenglish{\renewcommand{\contentsname}{Contents:}}

\usepackage{sphinxmessages}
\setcounter{tocdepth}{1}



\title{stlutils}
\date{Apr 08, 2024}
\release{0.0.1}
\author{Christian Mueller}
\newcommand{\sphinxlogo}{\vbox{}}
\renewcommand{\releasename}{Release}
\makeindex
\begin{document}

\pagestyle{empty}
\sphinxmaketitle
\pagestyle{plain}
\sphinxtableofcontents
\pagestyle{normal}
\phantomsection\label{\detokenize{index::doc}}



\chapter{cmutils}
\label{\detokenize{modules:cmutils}}\label{\detokenize{modules::doc}}

\section{stlutils module}
\label{\detokenize{stlutils:module-stlutils}}\label{\detokenize{stlutils:stlutils-module}}\label{\detokenize{stlutils::doc}}\index{module@\spxentry{module}!stlutils@\spxentry{stlutils}}\index{stlutils@\spxentry{stlutils}!module@\spxentry{module}}

\subsection{STL file generation}
\label{\detokenize{stlutils:stl-file-generation}}
\sphinxAtStartPar
This is a simple “brute force” conversion resulting in large files.
Each rectangle in the rastered data is simply converted to two triangles.

\sphinxAtStartPar
The resulting stl file therefore includes:  
\sphinxhyphen{} vertices: x\sphinxhyphen{},y\sphinxhyphen{} and z\sphinxhyphen{}coordinates of all points in the original surface  
\sphinxhyphen{} faces: triangles represented by the indices of their corner points  
\sphinxhyphen{} vectors: normals of triangle faces

\sphinxAtStartPar
You can reduce the size of the file afterwards in MeshLab using
“Filters \sphinxhyphen{}\textgreater{} Simplification: Quadratic Edge Collapse Decimation”,
where “Percentage reduction” is the target mesh size relative to original.
\index{add\_border() (in module stlutils)@\spxentry{add\_border()}\spxextra{in module stlutils}}

\begin{fulllineitems}
\phantomsection\label{\detokenize{stlutils:stlutils.add_border}}\pysiglinewithargsret{\sphinxcode{\sphinxupquote{stlutils.}}\sphinxbfcode{\sphinxupquote{add\_border}}}{\emph{\DUrole{n}{array}}, \emph{\DUrole{n}{border}}, \emph{\DUrole{n}{flip}}}{}
\sphinxAtStartPar
add border around 2D height array assuming uniform lattice spacing.
\begin{quote}\begin{description}
\item[{Parameters}] \leavevmode\begin{itemize}
\item {} 
\sphinxAtStartPar
\sphinxstyleliteralstrong{\sphinxupquote{array}} (\sphinxstyleliteralemphasis{\sphinxupquote{np.ndarray}}) \textendash{} array of shape (nx,ny) containing z coordinates.

\item {} 
\sphinxAtStartPar
\sphinxstyleliteralstrong{\sphinxupquote{border}} (\sphinxstyleliteralemphasis{\sphinxupquote{int}}) \textendash{} width of the border in ‘pixels’.

\item {} 
\sphinxAtStartPar
\sphinxstyleliteralstrong{\sphinxupquote{flip}} (\sphinxstyleliteralemphasis{\sphinxupquote{bool}}) \textendash{} whether or not to flip the topography upside\sphinxhyphen{}down.

\end{itemize}

\item[{Returns}] \leavevmode
\sphinxAtStartPar
updated 2D array of shape (nx+2*border, ny+2*border).

\item[{Return type}] \leavevmode
\sphinxAtStartPar
np.ndarray

\end{description}\end{quote}

\end{fulllineitems}

\index{add\_foundation() (in module stlutils)@\spxentry{add\_foundation()}\spxextra{in module stlutils}}

\begin{fulllineitems}
\phantomsection\label{\detokenize{stlutils:stlutils.add_foundation}}\pysiglinewithargsret{\sphinxcode{\sphinxupquote{stlutils.}}\sphinxbfcode{\sphinxupquote{add\_foundation}}}{\emph{\DUrole{n}{vertex\_list}}}{}
\sphinxAtStartPar
add to vertex list the vertices representing the bottom foundation of the 3D model.
\begin{quote}\begin{description}
\item[{Parameters}] \leavevmode
\sphinxAtStartPar
\sphinxstyleliteralstrong{\sphinxupquote{vertex\_list}} (\sphinxstyleliteralemphasis{\sphinxupquote{np.ndarray}}) \textendash{} array of shape (nx*ny, 3).

\item[{Returns}] \leavevmode
\sphinxAtStartPar
updated vertex list of shape ((nx+2)*(ny+2), 3).

\item[{Return type}] \leavevmode
\sphinxAtStartPar
np.ndarray

\end{description}\end{quote}

\end{fulllineitems}

\index{convertArray() (in module stlutils)@\spxentry{convertArray()}\spxextra{in module stlutils}}

\begin{fulllineitems}
\phantomsection\label{\detokenize{stlutils:stlutils.convertArray}}\pysiglinewithargsret{\sphinxcode{\sphinxupquote{stlutils.}}\sphinxbfcode{\sphinxupquote{convertArray}}}{\emph{\DUrole{n}{array}}, \emph{\DUrole{n}{outpath}}, \emph{\DUrole{n}{Lx}\DUrole{o}{=}\DUrole{default_value}{1}}, \emph{\DUrole{n}{Ly}\DUrole{o}{=}\DUrole{default_value}{0}}, \emph{\DUrole{n}{norm}\DUrole{o}{=}\DUrole{default_value}{1}}, \emph{\DUrole{n}{flip}\DUrole{o}{=}\DUrole{default_value}{False}}, \emph{\DUrole{n}{foundation}\DUrole{o}{=}\DUrole{default_value}{True}}, \emph{\DUrole{n}{border}\DUrole{o}{=}\DUrole{default_value}{0}}}{}
\sphinxAtStartPar
create stl file from 2D numpy array
\begin{quote}\begin{description}
\item[{Parameters}] \leavevmode\begin{itemize}
\item {} 
\sphinxAtStartPar
\sphinxstyleliteralstrong{\sphinxupquote{array}} (\sphinxstyleliteralemphasis{\sphinxupquote{np.ndarray}}) \textendash{} array of shape (nx,ny) containing z coordinates.

\item {} 
\sphinxAtStartPar
\sphinxstyleliteralstrong{\sphinxupquote{outpath}} (\sphinxstyleliteralemphasis{\sphinxupquote{str}}) \textendash{} filepath to the stl output file.

\item {} 
\sphinxAtStartPar
\sphinxstyleliteralstrong{\sphinxupquote{Lx}} (\sphinxstyleliteralemphasis{\sphinxupquote{float}}) \textendash{} physical dimension of the topography in x direction.

\item {} 
\sphinxAtStartPar
\sphinxstyleliteralstrong{\sphinxupquote{Ly}} (\sphinxstyleliteralemphasis{\sphinxupquote{float}}) \textendash{} physical dimension of the topography in y direction.

\item {} 
\sphinxAtStartPar
\sphinxstyleliteralstrong{\sphinxupquote{norm}} (\sphinxstyleliteralemphasis{\sphinxupquote{float}}) \textendash{} factor, by which to multiply z coordinates.

\item {} 
\sphinxAtStartPar
\sphinxstyleliteralstrong{\sphinxupquote{flip}} (\sphinxstyleliteralemphasis{\sphinxupquote{bool}}) \textendash{} whether or not to flip the topography upside\sphinxhyphen{}down.

\item {} 
\sphinxAtStartPar
\sphinxstyleliteralstrong{\sphinxupquote{foundation}} (\sphinxstyleliteralemphasis{\sphinxupquote{bool}}) \textendash{} whether or not to add 2 triangles representing the bottom of the 3D model

\item {} 
\sphinxAtStartPar
\sphinxstyleliteralstrong{\sphinxupquote{border}} (\sphinxstyleliteralemphasis{\sphinxupquote{int}}) \textendash{} width of the border in ‘pixels’.

\end{itemize}

\end{description}\end{quote}

\end{fulllineitems}

\index{convertFile() (in module stlutils)@\spxentry{convertFile()}\spxextra{in module stlutils}}

\begin{fulllineitems}
\phantomsection\label{\detokenize{stlutils:stlutils.convertFile}}\pysiglinewithargsret{\sphinxcode{\sphinxupquote{stlutils.}}\sphinxbfcode{\sphinxupquote{convertFile}}}{\emph{\DUrole{n}{inpath}}, \emph{\DUrole{n}{outpath}\DUrole{o}{=}\DUrole{default_value}{\textquotesingle{}\textquotesingle{}}}, \emph{\DUrole{n}{norm}\DUrole{o}{=}\DUrole{default_value}{1}}, \emph{\DUrole{n}{flip}\DUrole{o}{=}\DUrole{default_value}{True}}, \emph{\DUrole{n}{foundation}\DUrole{o}{=}\DUrole{default_value}{True}}}{}
\sphinxAtStartPar
create stl file from config file
\begin{quote}\begin{description}
\item[{Parameters}] \leavevmode\begin{itemize}
\item {} 
\sphinxAtStartPar
\sphinxstyleliteralstrong{\sphinxupquote{inpath}} (\sphinxstyleliteralemphasis{\sphinxupquote{str}}) \textendash{} filepath to a contMech config file containing nx*ny points.

\item {} 
\sphinxAtStartPar
\sphinxstyleliteralstrong{\sphinxupquote{outpath}} (\sphinxstyleliteralemphasis{\sphinxupquote{str}}) \textendash{} filepath to the stl output file.

\item {} 
\sphinxAtStartPar
\sphinxstyleliteralstrong{\sphinxupquote{norm}} (\sphinxstyleliteralemphasis{\sphinxupquote{float}}) \textendash{} factor, by which to multiply z coordinates.

\item {} 
\sphinxAtStartPar
\sphinxstyleliteralstrong{\sphinxupquote{flip}} (\sphinxstyleliteralemphasis{\sphinxupquote{bool}}) \textendash{} whether or not to flip the topography upside\sphinxhyphen{}down.

\item {} 
\sphinxAtStartPar
\sphinxstyleliteralstrong{\sphinxupquote{foundation}} (\sphinxstyleliteralemphasis{\sphinxupquote{bool}}) \textendash{} whether or not to add 2 triangles representing the bottom of the 3D model

\end{itemize}

\end{description}\end{quote}

\end{fulllineitems}

\index{create\_faces() (in module stlutils)@\spxentry{create\_faces()}\spxextra{in module stlutils}}

\begin{fulllineitems}
\phantomsection\label{\detokenize{stlutils:stlutils.create_faces}}\pysiglinewithargsret{\sphinxcode{\sphinxupquote{stlutils.}}\sphinxbfcode{\sphinxupquote{create\_faces}}}{\emph{\DUrole{n}{foundation}\DUrole{o}{=}\DUrole{default_value}{True}}}{}
\sphinxAtStartPar
calculate the 2*(nx\sphinxhyphen{}1)*(ny\sphinxhyphen{}1) triangles contained in a nx*ny surface
\begin{quote}\begin{description}
\item[{Parameters}] \leavevmode
\sphinxAtStartPar
\sphinxstyleliteralstrong{\sphinxupquote{foundation}} (\sphinxstyleliteralemphasis{\sphinxupquote{bool}}) \textendash{} whether or not to add 2 triangles representing the bottom of the 3D model

\item[{Returns}] \leavevmode
\sphinxAtStartPar
array of shape (2*(nx\sphinxhyphen{}1)*(ny\sphinxhyphen{}1) + foundation*2, 3) containing the indices of the 3 vertices of each triangle.

\item[{Return type}] \leavevmode
\sphinxAtStartPar
np.ndarray

\end{description}\end{quote}

\end{fulllineitems}

\index{from\_array() (in module stlutils)@\spxentry{from\_array()}\spxextra{in module stlutils}}

\begin{fulllineitems}
\phantomsection\label{\detokenize{stlutils:stlutils.from_array}}\pysiglinewithargsret{\sphinxcode{\sphinxupquote{stlutils.}}\sphinxbfcode{\sphinxupquote{from\_array}}}{\emph{\DUrole{n}{array}}, \emph{\DUrole{n}{Lx}\DUrole{o}{=}\DUrole{default_value}{1}}, \emph{\DUrole{n}{Ly}\DUrole{o}{=}\DUrole{default_value}{1}}, \emph{\DUrole{n}{norm}\DUrole{o}{=}\DUrole{default_value}{1}}, \emph{\DUrole{n}{flip}\DUrole{o}{=}\DUrole{default_value}{False}}}{}
\sphinxAtStartPar
convert 2D height topography to a 3D vertex list assuming uniform lattice spacing.
\begin{quote}\begin{description}
\item[{Parameters}] \leavevmode\begin{itemize}
\item {} 
\sphinxAtStartPar
\sphinxstyleliteralstrong{\sphinxupquote{array}} (\sphinxstyleliteralemphasis{\sphinxupquote{np.ndarray}}) \textendash{} array of shape (nx,ny) containing z coordinates.

\item {} 
\sphinxAtStartPar
\sphinxstyleliteralstrong{\sphinxupquote{Lx}} (\sphinxstyleliteralemphasis{\sphinxupquote{float}}) \textendash{} physical dimension of the topography in x direction.

\item {} 
\sphinxAtStartPar
\sphinxstyleliteralstrong{\sphinxupquote{Ly}} (\sphinxstyleliteralemphasis{\sphinxupquote{float}}) \textendash{} physical dimension of the topography in y direction.

\item {} 
\sphinxAtStartPar
\sphinxstyleliteralstrong{\sphinxupquote{flip}} (\sphinxstyleliteralemphasis{\sphinxupquote{bool}}) \textendash{} whether or not to flip the topography upside\sphinxhyphen{}down.

\end{itemize}

\item[{Returns}] \leavevmode
\sphinxAtStartPar
3D vertex positions as an array of shape (nx*ny, 3).

\item[{Return type}] \leavevmode
\sphinxAtStartPar
np.ndarray

\end{description}\end{quote}

\end{fulllineitems}

\index{from\_file() (in module stlutils)@\spxentry{from\_file()}\spxextra{in module stlutils}}

\begin{fulllineitems}
\phantomsection\label{\detokenize{stlutils:stlutils.from_file}}\pysiglinewithargsret{\sphinxcode{\sphinxupquote{stlutils.}}\sphinxbfcode{\sphinxupquote{from\_file}}}{\emph{\DUrole{n}{inpath}}, \emph{\DUrole{n}{norm}\DUrole{o}{=}\DUrole{default_value}{1}}, \emph{\DUrole{n}{flip}\DUrole{o}{=}\DUrole{default_value}{True}}}{}
\sphinxAtStartPar
convert 2D height topography to a 3D vertex list assuming uniform lattice spacing.
\begin{quote}\begin{description}
\item[{Parameters}] \leavevmode\begin{itemize}
\item {} 
\sphinxAtStartPar
\sphinxstyleliteralstrong{\sphinxupquote{inpath}} (\sphinxstyleliteralemphasis{\sphinxupquote{str}}) \textendash{} filepath to a contMech config file containing nx*ny points.

\item {} 
\sphinxAtStartPar
\sphinxstyleliteralstrong{\sphinxupquote{norm}} (\sphinxstyleliteralemphasis{\sphinxupquote{float}}) \textendash{} factor, by which to multiply z coordinates.

\item {} 
\sphinxAtStartPar
\sphinxstyleliteralstrong{\sphinxupquote{flip}} (\sphinxstyleliteralemphasis{\sphinxupquote{bool}}) \textendash{} whether or not to flip the topography upside\sphinxhyphen{}down.

\end{itemize}

\item[{Returns}] \leavevmode
\sphinxAtStartPar
3D vertex positions as an array of shape (nx*ny, 3).

\item[{Return type}] \leavevmode
\sphinxAtStartPar
np.ndarray

\end{description}\end{quote}

\begin{sphinxadmonition}{warning}{Warning:}
\sphinxAtStartPar
These files assume that the surface is periodically repeatable!
\end{sphinxadmonition}

\begin{sphinxadmonition}{warning}{Warning:}
\sphinxAtStartPar
flip=True is default since these files store the surface upside down!
\end{sphinxadmonition}

\end{fulllineitems}

\index{save\_mesh() (in module stlutils)@\spxentry{save\_mesh()}\spxextra{in module stlutils}}

\begin{fulllineitems}
\phantomsection\label{\detokenize{stlutils:stlutils.save_mesh}}\pysiglinewithargsret{\sphinxcode{\sphinxupquote{stlutils.}}\sphinxbfcode{\sphinxupquote{save\_mesh}}}{\emph{\DUrole{n}{vertex\_list}}, \emph{\DUrole{n}{faces\_list}}, \emph{\DUrole{n}{outpath}}}{}
\sphinxAtStartPar
create mesh from vertices and faces and save it to an stl file
\begin{quote}\begin{description}
\item[{Parameters}] \leavevmode\begin{itemize}
\item {} 
\sphinxAtStartPar
\sphinxstyleliteralstrong{\sphinxupquote{vertex\_list}} (\sphinxstyleliteralemphasis{\sphinxupquote{np.ndarray}}) \textendash{} array of shape (nx*ny, 3).

\item {} 
\sphinxAtStartPar
\sphinxstyleliteralstrong{\sphinxupquote{faces\_list}} (\sphinxstyleliteralemphasis{\sphinxupquote{np.ndarray}}) \textendash{} array of 3\sphinxhyphen{}tuples of indices, where each of those 3\sphinxhyphen{}tuples forms a triangle in vertex\_list.

\item {} 
\sphinxAtStartPar
\sphinxstyleliteralstrong{\sphinxupquote{outpath}} (\sphinxstyleliteralemphasis{\sphinxupquote{str}}) \textendash{} filepath to the stl output file.

\end{itemize}

\end{description}\end{quote}

\end{fulllineitems}



\chapter{Indices and tables}
\label{\detokenize{index:indices-and-tables}}\begin{itemize}
\item {} 
\sphinxAtStartPar
\DUrole{xref,std,std-ref}{genindex}

\item {} 
\sphinxAtStartPar
\DUrole{xref,std,std-ref}{modindex}

\item {} 
\sphinxAtStartPar
\DUrole{xref,std,std-ref}{search}

\end{itemize}


\renewcommand{\indexname}{Python Module Index}
\begin{sphinxtheindex}
\let\bigletter\sphinxstyleindexlettergroup
\bigletter{s}
\item\relax\sphinxstyleindexentry{stlutils}\sphinxstyleindexpageref{stlutils:\detokenize{module-stlutils}}
\end{sphinxtheindex}

\renewcommand{\indexname}{Index}
\printindex
\end{document}